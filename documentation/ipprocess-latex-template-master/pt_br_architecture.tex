%
% Portuguese-BR vertion
% 
\documentclass{report}

\usepackage{ipprocess}
% Use longtable if you want big tables to split over multiple pages.
\usepackage{longtable}
\usepackage{tikz}
\usepackage[utf8]{inputenc} 
\usepackage[brazil]{babel} % Uncomment for portuguese
\usepackage{siunitx}
\usepackage{pdflscape}

\sisetup{math-micro=\text{µ},text-micro=µ}


\sloppy

\graphicspath{{./pictures/}} % Pictures dir
\makeindex
\begin{document}

\DocumentTitle{Documento de Arquitetura}
\Project{Microarquitetura}
\Organization{Chip over the Hill}
\Version{Build 2.0}

\capa
\newpage
\newpage

%%%%%%%%%%%%%%%%%%%%%%%%%%%%%%%%%%%%%%%%%%%%%%%%%%
%% Revision History
%%%%%%%%%%%%%%%%%%%%%%%%%%%%%%%%%%%%%%%%%%%%%%%%%%
\chapter*{Histórico de Revisões}
\begin{center}
	\begin{longtable}[pos]{|m{2cm} | m{8cm} | m{4cm}|} 
		\hline
		\cellcolor[gray]{0.9}
		\textbf{Date} & \cellcolor[gray]{0.9}\textbf{Descrição} & \cellcolor[gray]{0.9}\textbf{Autor(s)}\\ \hline
		\endfirsthead
		\hline
		\multicolumn{3}{|l|}%
		{{\bfseries continuação da página anterior}} \\
		\hline
		\cellcolor[gray]{0.9}
		\textbf{Date} & \cellcolor[gray]{0.9}\textbf{Descrição} & \cellcolor[gray]{0.9}\textbf{Autor(s)}\\ \hline
		\endhead
		
		\multicolumn{3}{|r|}{{continua na próxima página}} \\ \hline
		\endfoot
		
		\hline
		\endlastfoot
		
      14/12/2015 &  Estruturação do documento & Patricia Gomes \\ \hline  
      15/12/2015 &  Finalização do documento & Patricia Gomes \\ \hline  
      16/12/2015 &  Revisão do documento & Fábio Barros e Matheus Borges \\ \hline 
      

    \end{longtable}
\end{center}

% TOC instantiation
\tableofcontents

%%%%%%%%%%%%%%%%%%%%%%%%%%%%%%%%%%%%%%%%%%%%%%%%%%
%% Document main content
%%%%%%%%%%%%%%%%%%%%%%%%%%%%%%%%%%%%%%%%%%%%%%%%%%
\newpage
\chapter{Introdução}
\section{Objetivo} 
O objetivo deste documento é definir as especificações do processador desenvolvido. As seções a seguir definem os parâmetros de implementação que compõem os requisitos gerais do processador.Tais requisitos incluem: arquitetura do conjunto de instruções, definições de entrada e saída e a arquitetura geral do processador. 

\section{Organização do Documento} 
Sessão 2: Apresenta uma visão geral da arquitetura e dos principais requisitos. \newline
Sessão 3: Especifica o conjunto de instruções do processador.

 % inicio da tabela de acronimos e abreviacoes do documento
 \section{Acrônimos e Abreviações}
   \FloatBarrier
    \begin{table}[H]
      \begin{center}
        \begin{tabular}[pos]{|m{2cm} | m{12cm}|} 
          \hline
          \cellcolor[gray]{0.9}\textbf{Sigla} & \cellcolor[gray]{0.9}\textbf{Descrição} \\ \hline
             UC      &  Unidade de Controle\\ \hline
             GPR   & {Registrador de Propósito Geral}\\ \hline
        \end{tabular}
      \end{center}
    \end{table}  
  % fim
  
  
\newpage
\chapter{Visão Geral da Arquitetura}
O projeto tem como objetivo a implementação de um processador de 32 bits. O processador de arquitetura de 32 bits, possui 16 GPR de 32 bits de largura (r0 ... r15) e ISA composta por 42 instruções.
E Cada instrução executada nesta arqutetura demora quatro ciclos para completar. Esses ciclos são denominados IF, ID, EX e WB, como descrito na Figura abaixo. No primeiro ciclo (IF), a próxima instrução a ser executada é carregada da memória e armazenada no registrador de instruções (IR). No segundo ciclo (ID), a instrução é decodificada e os operandos são lidos do banco de registradores. No terceiro ciclo (EX), a instrução é executada e as condições dos resultados  são calculadas. No quarto ciclo (WB), os resultados são escritos no banco de registradores.

\begin{figure}[H]
	\centering
	\includegraphics[width=\textwidth]{./pictures/Datapath.png}
	\caption{Datapath Geral}
\end{figure}

\section{Principais características}
\begin{itemize}
    \item \textbf Arquitetura de 32 bits;
    \item \textbf 16 GPR de 32 bits de largura (r0 ... r15);
    \item \textbf ISA composta por 42 instruções;
    \item \textbf Instruções de 3 operandos;
    \item \textbf O projeto foi descrito em C;
    \item \textbf Um conjunto de programas de teste em assembly foi implementado,objetivando validar a implementação.
\end{itemize}

\newpage
\chapter{Arquitetura das Instruções}
O conjunto de instruções do processador foi dividido nos seguintes grupos:\newline
\begin{itemize}
 \item \textbf Instruções lógicas e aritméticas;
 \item \textbf Instruções com constante;
 \item \textbf Instruções de acesso à memória;
 \item \textbf Instruções de desvio;
 \item \textbf Instruções de desvio por registrador;
 \item \textbf NOP;
 \item \textbf HALT.
\end{itemize}

\section{Instruções Lógicas e Aritméticas}
As instruções de lógica e aritmética possuem so seguinte formato:

% inicio da tabela de formato de instruções de lógica e aritmética
   \FloatBarrier
    \begin{table}[H]
      \begin{center}
        \begin{tabular}[pos]{|m{2cm}|m{2cm}|m{2cm}|m{2cm}|m{2cm}|m{4cm}|} 
          \hline
          \cellcolor[gray]{0.9}\textbf{31:29} & \cellcolor[gray]{0.9}\textbf{28:24} &
          \cellcolor[gray]{0.9}\textbf{23:20} &
          \cellcolor[gray]{0.9}\textbf{19:16} &
          \cellcolor[gray]{0.9}\textbf{15:12} &
          \cellcolor[gray]{0.9}\textbf{11:0} \\ \hline
            0 0 1 & OP & WC & RA & RB & X X X X X X X X X X X X \\ \hline
        \end{tabular}
      \end{center}
    \end{table}  
  % fim


% inicio da tabela de operações das instruções de lógica e aritmética
   \FloatBarrier
    \begin{table}[H]
      \begin{center}
        \begin{tabular}[pos]{|m{3,5cm}|m{3,5cm}|m{4cm}|m{4cm}|} 
          \hline
          \cellcolor[gray]{0.9}\textbf{OPULA} & 
          \cellcolor[gray]{0.9}\textbf{Mnemônico} &
          \cellcolor[gray]{0.9}\textbf{Operação} &
          \cellcolor[gray]{0.9}\textbf{Flags Atualizadas} \\ \hline
            00000 & add c,a,b & C = A + B & O S C Z \\ \hline
            00001 & addinc c,a,b & C = A + B + 1 & O S C Z \\ \hline
            00011 & inca c,a & C = A + 1 & O S C Z \\ \hline
            00100 & subdec c,a,b & C = A – B – 1 & O S C Z \\ \hline
            00101 & sub c, a, b & C = A – B & O S C Z \\ \hline
            00111 & deca c, a & C = A – 1 & O S C Z \\ \hline
            01000 & lsl c, a &  C = Deslocamento Lógico Esq. (A) & S C Z \\ \hline
            01001 & asr c, a & C = Deslocamento Aritmético Dir. (A) & O S C Z \\ \hline
            10000 & zeros c & C = 0 & Z \\ \hline
            10001 & and c, a, b & C = AandB & SZ \\ \hline
            10010 & andnota c,a,b & C = !AandB & S Z \\ \hline
            10011 & passb c, b & C = B & nenhuma \\ \hline
            10100 & andnotb c, a, b & C = Aand!B & S Z \\ \hline
            10101 & passa, c, a & C = A & S Z \\ \hline
            10110 & xor c, a, b & C = A xor B & S Z \\ \hline
            10111 & or c, a, b & C = A | B & S Z \\ \hline
            11000 & nand c, a, b & C = !Aand!B &  S Z \\ \hline
            11001 & xnor c, a, b & C = !(A xor B) &  S Z \\ \hline
            11010 & passnota c, a & C = !A &  S Z \\ \hline
            11011 & ornota c, a, b & C = !A|B &  S Z \\ \hline
            11100 & passnotb c, b & C = !B &  S Z \\ \hline
            11101 & ornotb c, a, b & C = A|!B &  S Z \\ \hline
            11110 & nor c, a, b & C = !A|!B &  S Z \\ \hline
            11111 & ones c & C = 1 &  nenhuma \\ \hline
        \end{tabular}
      \end{center}
    \end{table}  
  % fim


\section{Instruções com Constante}
As instruções com constante possuem so seguinte formato:

% inicio da tabela de formato de instruções com constante
   \FloatBarrier
    \begin{table}[H]
      \begin{center}
        \begin{tabular}[pos]{|m{2cm}|m{2cm}|m{2cm}|m{2cm}|m{2cm}|m{5cm}|} 
          \hline
          \cellcolor[gray]{0.9}\textbf{31:29} & \cellcolor[gray]{0.9}\textbf{28:26} &
          \cellcolor[gray]{0.9}\textbf{25:24} &
          \cellcolor[gray]{0.9}\textbf{23:20} &
          \cellcolor[gray]{0.9}\textbf{19:16} &
          \cellcolor[gray]{0.9}\textbf{15:0} \\ \hline
            0 1 0 & X X X & OP & WC & X X X X X X & CONSTANTE \\ \hline
        \end{tabular}
      \end{center}
    \end{table}  
  % fim

% inicio da tabela de operações das instruções com constante
   \FloatBarrier
    \begin{table}[H]
      \begin{center}
        \begin{tabular}[pos]{|m{3,5cm}|m{3,5cm}|m{8cm}|}  
          \hline
          \cellcolor[gray]{0.9}\textbf{OPCT} & 
          \cellcolor[gray]{0.9}\textbf{Mnemônico} &
          \cellcolor[gray]{0.9}\textbf{Operação} \\ \hline
            01 & lcl c, Const16 & C = Const16|(Cand0x00) \\ \hline
            10 & loadlit c, Const16 & C = (Const16 « 16)|(Cand0x00ff ) \\ \hline
            11 & lch c, Const16 & C = CONSTANTE \\ \hline
        \end{tabular}
      \end{center}
    \end{table}  
  % fim
  
  
\section{Instruções de Acesso à Memória}
As instruções de acesso à memória possuem so seguinte formato:

% inicio da tabela de formato de instruções de acesso à memória
   \FloatBarrier
    \begin{table}[H]
      \begin{center}
        \begin{tabular}[pos]{|m{2cm}|m{2cm}|m{1cm}|m{1cm}|m{1cm}|m{1cm}|m{6cm}|} 
          \hline
          \cellcolor[gray]{0.9}\textbf{31:29} & 
          \cellcolor[gray]{0.9}\textbf{28:25} &
          \cellcolor[gray]{0.9}\textbf{24} &
          \cellcolor[gray]{0.9}\textbf{23:20} &
          \cellcolor[gray]{0.9}\textbf{19:16} &
          \cellcolor[gray]{0.9}\textbf{15:12} & 
          \cellcolor[gray]{0.9}\textbf{11:0} \\ \hline
            1 0 0 & X X X X & OP & WC & RA & RB &  X X X X X X X X X X X X \\ \hline
        \end{tabular}
      \end{center}
    \end{table}  
  % fim

% inicio da tabela de operações das instruções de acesso à memória
   \FloatBarrier
    \begin{table}[H]
      \begin{center}
        \begin{tabular}[pos]{|m{4cm}|m{4cm}|m{8cm}|} 
          \hline
          \cellcolor[gray]{0.9}\textbf{OPM} & 
          \cellcolor[gray]{0.9}\textbf{Mnemônico} &
          \cellcolor[gray]{0.9}\textbf{Operação} \\ \hline
            0 & load c, a & C = Mem[A] \\ \hline
            1 & store a, b & Mem[A] = B \\ \hline
        \end{tabular}
      \end{center}
    \end{table}  
  % fim
  
  
\section{Instruções de Desvio}
As instruções de desvio possuem so seguinte formato:

% inicio da tabela de formato de instruções de desvio
   \FloatBarrier
    \begin{table}[H]
      \begin{center}
        \begin{tabular}[pos]{|m{2cm}|m{2cm}|m{1cm}|m{2cm}|m{5cm}|m{2cm}|} 
          \hline
          \cellcolor[gray]{0.9}\textbf{31:29} & 
          \cellcolor[gray]{0.9}\textbf{28:27} &
          \cellcolor[gray]{0.9}\textbf{26:24} &
          \cellcolor[gray]{0.9}\textbf{23:21} &
          \cellcolor[gray]{0.9}\textbf{21:12} &
          \cellcolor[gray]{0.9}\textbf{11:0} \\ \hline
            1 0 1 & X X & OP & COND & X X X X X X X X & DESTINO \\ \hline
        \end{tabular}
      \end{center}
    \end{table}  
  % fim

% inicio da tabela de operações das instruções de desvio
   \FloatBarrier
    \begin{table}[H]
      \begin{center}
        \begin{tabular}[pos]{|m{4cm}|m{4cm}|m{8cm}|} 
          \hline
          \cellcolor[gray]{0.9}\textbf{OPD} & 
          \cellcolor[gray]{0.9}\textbf{Mnemônico} &
          \cellcolor[gray]{0.9}\textbf{Operação} \\ \hline
            00 & j DESTINO & Jump Incondicional \\ \hline
            01 & jt.cond DESTINO & Jump True \\ \hline
            10 & jf.cond DESTINO & Jump False \\ \hline
        \end{tabular}
      \end{center}
    \end{table}  
  % fim
  
  
\subsection{Tabela de Condições para os Desvios}
As instruções de desvio condicional devem testar as condições apresentadas no quadro abaixo:

% inicio da tabela de condições
   \FloatBarrier
    \begin{table}[H]
      \begin{center}
        \begin{tabular}[pos]{|m{2cm}|m{3cm}|m{8cm}|} 
          \hline
          \cellcolor[gray]{0.9}\textbf{COND} & 
          \cellcolor[gray]{0.9}\textbf{Mnemônico} &
          \cellcolor[gray]{0.9}\textbf{Condição}  \\ \hline
            001 & neg & Resultado da ALU negativo \\ \hline
            010 & zero & Resultado da ALU zero \\ \hline
            100 & carry & Carry da ALU \\ \hline
            101 & negzero & Resultado da ALU negativo ou zero \\ \hline
            110 & true & TRUE \\ \hline
            111 & overflow & Resultado da ALU overflow \\ \hline
        \end{tabular}
      \end{center}
    \end{table}  
  % fim




\section{Instruções de Desvio por Registrador}
As instruções de desvio por registrador possuem so seguinte formato:

% inicio da tabela de formato de instruções de desvio por registrador
   \FloatBarrier
    \begin{table}[H]
      \begin{center}
        \begin{tabular}[pos]{|m{2cm}|m{2cm}|m{1cm}|m{3cm}|m{1cm}|m{5cm}|} 
          \hline
          \cellcolor[gray]{0.9}\textbf{31:29} & 
          \cellcolor[gray]{0.9}\textbf{28:25} &
          \cellcolor[gray]{0.9}\textbf{24} &
          \cellcolor[gray]{0.9}\textbf{23:16} &
          \cellcolor[gray]{0.9}\textbf{15:12} &
          \cellcolor[gray]{0.9}\textbf{11:0} \\ \hline
            1 1 0 & X X X X & OP & X X X X X X X X & RB & X X X X X X X X X X X X \\ \hline
        \end{tabular}
      \end{center}
    \end{table}  
  % fim

% inicio da tabela de operações das instruções de desvio por registrador
   \FloatBarrier
    \begin{table}[H]
      \begin{center}
        \begin{tabular}[pos]{|m{4cm}|m{4cm}|m{7cm}|} 
          \hline
          \cellcolor[gray]{0.9}\textbf{OPDR} & 
          \cellcolor[gray]{0.9}\textbf{Mnemônico} &
          \cellcolor[gray]{0.9}\textbf{Operação} \\ \hline
            0 & jal b & Jump and Link \\ \hline
            1 & jr b & Jump Register \\ \hline
        \end{tabular}
      \end{center}
    \end{table}  
  % fim
  
  
\section{HALT}
O HALT possui o formato seguinte:

% inicio da tabela de formato do HALT
    \FloatBarrier
    \begin{table}[H]
      \begin{center}
        \begin{tabular}[pos]{|m{2cm}|m{2cm}|m{1cm}|m{3cm}|m{1cm}|m{5cm}|} 
          \hline
          \cellcolor[gray]{0.9}\textbf{31:29} & 
          \cellcolor[gray]{0.9}\textbf{28:27} &
          \cellcolor[gray]{0.9}\textbf{26:24} &
          \cellcolor[gray]{0.9}\textbf{23:12} &
          \cellcolor[gray]{0.9}\textbf{11:0}  \\ \hline
            1 0 1 & X X & 00 & X X X X X X X X & L \\ \hline
        \end{tabular}
      \end{center}
    \end{table}  
  % fim

% inicio da tabela de operação HALT
   \FloatBarrier
    \begin{table}[H]
      \begin{center}
        \begin{tabular}[pos]{|m{4cm}|m{4cm}|m{7cm}|} 
          \hline
          \cellcolor[gray]{0.9}\textbf{Mnemônico} &
          \cellcolor[gray]{0.9}\textbf{Operação} \\ \hline
            L: j L & Jump Here \\ \hline
        \end{tabular}
      \end{center}
    \end{table}  
  % fim
  
\section{NOP}
A NOP possui o formato seguinte:

% inicio da tabela de formato do NOP
   \FloatBarrier
    \begin{table}[H]
      \begin{center}
        \begin{tabular}[pos]{|m{2cm}|m{14cm}|} 
          \hline
          \cellcolor[gray]{0.9}\textbf{31:29} & 
          \cellcolor[gray]{0.9}\textbf{28:0} \\ \hline
            0 0 0 & X X X X X X X X X X X X X X X X X X X X X X X X X X X \\ \hline
        \end{tabular}
      \end{center}
    \end{table}  
  % fim
  
\chapter{Descrição dos Componentes}

Durante algumas discussões foi discutido quais  componentes constituiriam o processador. A partir de análises das instruções foram listados os componentes a seguir:\newline
\begin{itemize}
 \item \textbf PC – O Contador de Programa (PC) é o registrador que armazena o endereço da próxima instrução a ser executada. 
 
 \item \textbf Memória de Instruções –  Tem como funcionalidade armazenar a instrução a ser executada. 
 \item \textbf Banco de Registradores – O banco de registradores consiste em 16 registradores de propósito geral de 32 bits. 
 
 \item \textbf Extensor de Sinal – O extensor de sinal é utilizado para extender o sinal dos bits de entrada nas operações com constantes.
 
 \item \textbf ULA – A Unidade Lógica e Aritmética (ULA) é um circuito combinacional responsável por realizar operações aritméticas e lógicas dentro de um processador. As operações a serem executadas são determinadas por meio dos sinais de controle das suas entradas de operação, logo após os dados de entrada são computados e o resultado é obtido na saída do circuito. 
 
 \item \textbf MemóriadeDados–A Memória de dados tem como propósito salvar/ler dados proveniente da instruções de acesso à memória. 
 
 \item \textbf Registrador de Flags – O Registrador de flags é responsável por armazenar os estados das flags Overflow, Carry, Sinal e Zero. Estes estados são atualizados de acordo com o resultado das operações efetuadas pela unidade lógica e aritmética. 
 
 \item \textbf Testador de Flags – Esse módulo foi criado com o objetivo de decidir se um jump será ou não realizado conforme as condições definidas. 
 
 \item \textbf Unidade de Controle – Esta unidade é responsável por gerar todos os sinais de controle do processador, como sinais de leitura, escrita de memoria e de  registradores de armazenamento temporário interno e sinais de liberação de barramentos para endereço e dados e unidades funcionais. As unidades funcionais internas do processador são controladas por esta unidade de temporização e controle. Estes sinais de controle são enviados para as unidades funcionais após a decodificação de uma instrução.
\end{itemize}

  


% Optional bibliography section
% To use bibliograpy, first provide the ipprocess.bib file on the root folder.
% \bibliographystyle{ieeetr}
% \bibliography{ipprocess}

\end{document}
